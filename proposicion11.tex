Sea $H$ un grupo cíclico y sea $N$ un grupo arbitrario. Si $\varphi$ and $\psi$ son monomorfismos 
de $H$ a $Aut(N)$ tales que $\varphi(H) = \psi(H)$, entonces 
$N \rtimes_\varphi H \cong N \rtimes_\psi N$. \par\null

\begin{proof}
    Sea $H = <x>$. Como las imágenes de $H$ bajo $\varphi$ y $\psi$, $\varphi(x)$ y $\psi(x)$ generan
    al mismo subgrupo cícilco de $Aut(N)$. Por lo tanto existen $a,b \in \Z$ tales que 
    $\varphi(x)^a = \psi(x)$ y $\varphi(x) = \psi(x)^b$. \\ 
    
    De aquí que:
    \begin{equation} 
      \varphi(x) = \psi(x)^b = \varphi(x)^{ab} = \varphi(x^{ab}).
    \end{equation}

    Como $\varphi$ es monomorfimso, tenemos que 
    \begin{equation} \label{propocicion11:identidad}
        x = x^{ab}
    \end{equation} 
    
    Es decir, elevar a la $ab$ es otra manera de escribir el homomorfismo identidad.		
    
    Como $H$ es cíclico tenemos que para todo $h \in H$, existe $r \in \Z$ tal que $x^r = h$ y entonces
    $$
    \varphi(h^a) = \varphi((x^r)^a) = 
    \varphi(x^ar) = (\varphi(x)^a)^r = 
    \psi(x)^r = \psi(x^r) = \psi(h),
    $$ 
    análogamente $\varphi(h) = \psi(h^b)$.  \\
    
    Definamos $\tau : N \rtimes_\varphi H \rightarrow N \rtimes_\psi N$ como $\tau(n,h)= (n, h^a)$.\\
    
    \begin{align}
    \tau((n_1, h_1)(n_2, h_2))	&=  \tau(n_1 \psi(h_1)(n_2), h_1h_2)\\
                                &= (n_1\psi(h1)(n_2), (h_1 h_2)^a) \\
                                &= (n_1\varphi(h_1^a)(n_2), h_1^a h_2^a)\\
                                &= n_1 h_1^a n2 h_2^a  \;\text{(por definición de} \; N \rtimes_\psi H \text{)}\\
                                &= \tau(n_1 h_1) \tau(n_2 h_2)
    \end{align}
                                    
    Con esto, tenemos que $\tau$ separa productos y manda inversos en inversos. Por lo tanto $\tau$ es
    homomorfismo. \\
    
    Análogamente $\lambda : N \rtimes_\psi H \rightarrow N \rtimes_\varphi N$
    definida como $\lambda(n,h) = (n, h^b)$, resulta ser homomorfismo.\\
        
    Ahora notemos que $\tau \circ \lambda(n,h) = \tau(n, h^b) = (n, h^{ab})$ y por 
    (\ref{propocicion11:identidad}), tenemos que $\tau \circ \lambda = id_{N \rtimes_\psi H}$.\\
    
    Análogamente, tenemos que $\lambda \circ \tau = id_{N \rtimes_\varphi H}$. Con esto tenemos que
    tanto $\tau$ como $\lambda$ son isomorfismos, con lo que termina la demostración. 
\end{proof}