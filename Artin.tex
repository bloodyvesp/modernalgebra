Comenzaremos por hacer las definiciones pertinentes.\par\null

\begin{definicion}
    Sea $Aut(E)$ el grupo de todos los automorfismos de un campo $E$. Si $G \subset Aut(E)$, 
    entonces a
    \begin{align}
        E^G = \{  \alpha \in E : \sigma(\alpha) = \alpha para todo \sigma \in G \}
    \end{align}
    se le llama el \textbf{campo fijo}.
\end{definicion}

Antes de pasar a la demostraciónde del teorema de Artin, enunciaremos sin demostración un lema que nos
será de utilidad.

\begin{lema}\label{Artin:lema}
    Si $G = \{ \sigma_1, \dots, \sigma_n \} \subset Aut(E)$, entonces
    \begin{align}
        [E:E^G]     &\geq   n
    \end{align}
\end{lema}\par\null

\begin{teorema}[Artin]
    Si $G=\{ \sigma_1, \dots, \sigma_n \} \leq Aut(E)$, entonces:
    \begin{align}
        [E : E^G]   =   \abs{G}
    \end{align}
\end{teorema}

\begin{proof}
    Gracias a [\ref{Artin:lema}] sólo hace falta demostrar que no es posible que $[E : E^G] > n$.
    Supongamos esto cierto.\par\null
        
    Esto significa que existe un conjunto con al menos $n+1$ elementos linealmente independientes en $E$ como
    $E^G$-espacio vectorial. Sean $\{ \omega_1, \dots, \omega_{n+1} \}$.\par\null
    
    
\end{proof}