Antes de comenzar con la demostración del Lema de Dedekind, haremos
algunas definiciones.

\begin{definicion} 
    Un \textbf{caracter} de un grupo $G$ en un campo $E$ es un homomofrphismo de grupos
    
    \begin{align}
            \sigma: G \longrightarrow E^{\#}
    \end{align}
    
    Donde $E^{\#} = E - \{0\}$ es el grupo multiplicativo de $E$.
\end{definicion}

\begin{definicion}
    Un conjunto $\{\sigma_1 , \sigma_2, \dots, \sigma_n\}$ de caracteres de un grupo $G$ en un campo $E$ es
    \textbf{independiente} si no existen $a_1, a_2, \dots, a_n \in E$, no todos 0, tales que 
    
    \begin{align}
            \sum a_i \sigma_i(x) = 0 \;\;\;\; \forall x \in G 
    \end{align}\par\null
    
    %\begin{sloppypar}
    %\tiny
        %\texttt{
                %[\textbf{Nota:} En el libro se define distinto. 
                %Escriben el $x \in G$ hasta el final. 
                %Definido así quedría decir que basta que exista un $x$
                %para el cual la suma nunca se hace $0$ para que el 
                %conjunto de caracteres es independiente.                
                %Es una sutileza del orden en el que se dicen las 
                %cosas y lo investigué en distintas fuentes 
                %para estar seguro.]}
    %\normalsize
    %\end{sloppypar}
\end{definicion}

Ahora estamos listos para el Lema de Dedekind.

\begin{lema}[Dedekind]
    Todo conjunto $\{\sigma_1, \dots, \sigma_n\}$ de caracteres distintos de un grupo $G$ en un campo $E$ es independiente.
\end{lema}

\begin{proof}
    Procedemos por inducción sobre $n$. 
    
    \begin{itemize}
        \item Base de inducción\par\null
            Sea $n = 1$. Si $\{\sigma_1\}$ no fuera independiente entonces existiría $a_1 \in E$, con $a_1 \not= 0$ 
            tal que $a_1 \sigma_1(x) = 0$. Pero tanto $a_1$ como $\sigma_1(x)$ son distintos de $0$ y pertenecen a un campo
            (es decir, a un dominio entero) y por lo tanto no es posible que $a_1 \sigma_1(x) = 0$.\par\null   
                    
         \item Hipótesis de inducción\par\null
            Sea $n>1$. Y supongamos que para $m < n$ se cumple el resultado.\par\null
                    
         \item Paso inductivo\par\null
            Supongamos que existen $a_1, \dots, a_n \in E$ tales que para todo $x \in G$ se tiene que
                  
            \begin{align}
                    \sum a_i \sigma_i(x) = 0 \label{dedekind:suma} 
            \end{align}\par\null
            
            Por hipótesis de inducción, tenemos que $a_1, \dots, a_n$ son necesariamente todos distintos de cero.
            También podemos suponer que $a_n = 1$, si no es así, basta multiplicar la suma por $a_n^{-1}$.\par\null
            
            Como $\sigma_n \not= \sigma_1$, necesariamente existe $y \in G$ tal que $\sigma_n(y) \not= \sigma_1(y)$.
            Como \eqref{dedekind:suma} aplica para todo $x \in G$, en particular aplica para $yx$. Entonces tenemos que
    
            \begin{align}
                    \sum a_i \sigma_i(yx)   &=  \sum a_i \sigma_i(y)\sigma_i(x) \\
                                            &=  0 
            \end{align}
            
            Multiplicando esto por $\sigma_n(y)^{-1}$ y, recordando que $a_n = 1$, obtenemos

            \begin{align}
                    \sigma_n(y)^{-1} \sum a_i \sigma_i(y)\sigma_i(x)    &=  \sum a_i \sigma_n(y)^{-1}\sigma_i(y)\sigma_i(x)                         \\
                                                                        &=  \sum_{i < n} a_i \sigma_n(y)^{-1}\sigma_i(y)\sigma_i(x) +   \sigma_n(x) \\
                                                                        &=  0. 
            \end{align}
            
            Restando esto último de \eqref{dedekind:suma} obtenemos
            
            \begin{align}
                    \sum_{i < n} a_i \sigma_i(x) + \sigma_n(x)  &-  (\sum_{i < n} a_i \sigma_n(y)^{-1}\sigma_i(y)\sigma_i(x) + \sigma_n(x))     \\ 
                                                                &=  \sum_{i < n} a_i \sigma_i(x) - a_i \sigma_n(y)^{-1}\sigma_i(y)\sigma_i(x)   \\
                                                                &=  \sum_{i < n} a_i (1 - \sigma_n(y)^{-1}\sigma_i(y)) \sigma_i(x)              \\
                                                                &=  0.                                                                                                                       
            \end{align}                                   
            
            Por un lado, la hipótesis de inducción nos dice que esto es cierto sólamente si 
            $a_i (1 - \sigma_n(y)^{-1}\sigma_i(y)) = 0$ para todo
            $i$. En particular para $i = 1$ tendríamos $a_1 (1 - \sigma_n(y)^{-1}\sigma_1(y)) = 0$. 
            Como $a_1 \not= 0$ esto obliga a $\sigma_n(y)^{-1}\sigma_1(y) = 1$, y por lo tanto 
            $\sigma_n(y) = \sigma_1(y)$, lo cual contradice nuestra elección de $y$.\par\null
            
            Y con esto hemos demostrado que no existen tales $a_1, \dots, a_n$.
    \end{itemize}
\end{proof}
