Antes de comenzar con la demostración del Lema de Dedekind, haremos
algunas definiciones.

\begin{definicion} 
    Un \textbf{caracter} de un grupo $G$ en un campo $E$ es un homomofrphismo de grupos
    
    \begin{align}
            \sigma: G \longrightarrow E^{\#}
    \end{align}
    
    Donde $E^{\#} = E - \{0\}$ es el grupo multiplicativo de $E$.
\end{definicion}

\begin{definicion}
    Un conjunto $\{\sigma_1 , \sigma_2, \dots, \sigma_n\}$ de caracteres de un grupo $G$ en un campo $E$ es
    \textbf{independiente} si para todo $x \in G$, no existen $a_1, a_2, \dots, a_n \in E$, no todos 0, tales que 
    
    \begin{align}
            \sum a_i \sigma_i{x} = 0 
    \end{align}\par\null
    
    \begin{sloppypar}
    \tiny
        \texttt{
                [\textbf{Nota:} En el libro se define distinto. 
                Escriben el $x \in G$ hasta el final. 
                Definido así quedría decir que basta que exista un $x$
                para el cual la suma nunca se hace $0$ para que el 
                conjunto de caracteres es independiente.                
                Es una sutileza del orden en el que se dicen las 
                cosas y lo investigué en distintas fuentes 
                para estar seguro.]}
    \normalsize
    \end{sloppypar}
\end{definicion}

Ahora estamos listos para el Lema de Dedekind.

\begin{lema}
    Todo conjunto $\{\sigma_1, \dots, \sigma_n\}$ de caracteres distintos de un grupo $G$ en un campo $E$ es independiente.
\end{lema}

\begin{proof}
    Procedemos por inducción sobre $n$. 
    
    \begin{itemize}
        \item Hipótesis de inducción\par\null
            Supongamos que el resultado es válido para cierto $n \geq 1$.\par\null
            
        \item Paso inductivo\par\null
        
        \item Base de inducción\par\null
            Sea $n = 1$. Si $\{\sigma_1\}$ no fuera independiente entonces existiría $x \in G$ y $a_1 \in E$, con $a_1 != 0$ 
            tal que $a_1 \sigma_1(x) = 0$. Pero tanto $a_1$ como $\sigma_1(x)$ son distintos de $0$ y pertenecen a un campo
            (es decir, a un dominio entero) y por lo tanto no es posible que $a_1 \sigma_1(x) = 0$.\par\null            
    \end{itemize}
\end{proof}
