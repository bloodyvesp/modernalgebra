\begin{definicion}
    Sea $E/F$ una extensi\'on de Galois y sea $\alpha\in E\setminus\{0\}$. 
    Definimos la norma $N(\alpha)$ de $\alpha$ por \[N(\alpha) := \prod_{\sigma \in Gal(E/F)}\sigma(\alpha).\]
\end{definicion}


\begin{teorema}[Teorema de Hilbert]
    Sea $E/F$ una extensi\'on de Galois cuyo grupo de Galois $G = Gal(E/F)$ es c\'iclico 
    de orden $n$. Sea $\sigma$ un generador de $G$. Entonces $N(\alpha) = 1$ si y s\'olo 
    si existe $\beta\in E\setminus \{0\}$ tal que \[\alpha = \beta\sigma (\beta)^{-1}.\]
\end{teorema}

\begin{proof}
    Si $\alpha = \beta\sigma (\beta)^{-1}$, entonces
    \begin{eqnarray*}
	N(\alpha) &	=&	N(\beta\sigma (\beta)^{-1})		\\
		  &	=&	N(\beta)N\big(\sigma(\beta)^{-1}\big)	\\
		  &	=&	N(\beta)N\big(\sigma(\beta)\big)^{-1}	\\
		  &	=&	N(\beta)N(\beta)^{-1}			\\
		  &	=&	1.
    \end{eqnarray*}

    Para probar la implicaci\'on contraria definamos la siguiente sucesi\'on
    \begin{eqnarray*}
	\delta_0 &		=&		\alpha,					\\
	\delta_1 &		=&		\alpha\sigma(\alpha),			\\
	\delta_2 &		=&		\alpha\sigma(\alpha)\sigma^{2}(\alpha),	\\
		 &		\vdots&							\\
	\delta_{n-1} &		=&		\alpha\sigma(\alpha)\sigma^{2}(\alpha)\cdots\sigma^{n-1}(\alpha).
    \end{eqnarray*}
    Dado que $\langle \sigma\rangle = Gal(E/F)$, tenemos que $\delta_{n-1} = N(\alpha) = 1$. Ahora, tenemos que para toda $i \in \{0,1,\ldots, n-2\}$ se tiene que
    \begin{equation}    \label{delta}
	\alpha\sigma(\delta_i) = \delta_{i+1}.
    \end{equation}
    Tenemos que los caracteres $\{1, \sigma, \sigma^2, \ldots, \sigma^{n-1}\}$ son independientes, por lo tanto existe $\gamma\in E$ tal que
    \[\delta_0\gamma + \delta_1\sigma(\gamma) + \delta_2\sigma^2(\gamma) + \cdots + \delta_{n-2}\sigma^{n-2}(\gamma) + \sigma^{n-1}(\gamma) = \beta \neq 0.\]
    Usando la ecuaci\'on \eqref{delta} se tiene que
    \[\sigma(\beta) = \alpha^{-1}\big[\delta_1\sigma(\gamma) + \delta_2\sigma^2(\gamma) + \cdots + \delta_{n-1}\sigma_{n-1}(\gamma)\big] + \sigma^n(\gamma).\]
    Como $\sigma^n = 1$, se tiene que el \'ultimo sumando es $\gamma = \alpha^{-1}\delta_0\gamma$. Factorizando $\alpha^{-1}$ tenemos que $\sigma(\beta) = \alpha^{-1}\beta$, por lo tanto $\alpha = \beta\sigma (\beta)^{-1}$.
\end{proof}
