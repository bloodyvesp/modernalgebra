Sean $N$ y $H$ grupos, sea $\psi : H \rightarrow Aut(N)$ un homomorfismo y $f \in Aut(N)$. 
Si $\hat{f}$ es el automorfismo interno de $Aut(N)$ inducido por $f$, entonces
$N \rtimes_{\hat{f} \circ \psi} \cong N \rtimes_\psi H$.\\

\begin{proof}
    Sea $\theta : N \rtimes_\psi H \rightarrow N \rtimes_{\hat{f} \circ \psi}$ definida por
    $\theta(n,h) = (f(n), h)$. Veamos que $\theta$ es homomorfismo:
    
    \begin{align}
        \theta((n_1, h_1)\cdot(n_2, h_2))&= \theta(n_1 \psi(h_1)(n_2), h_1 h_2) \\
                                         &= (f(n_1 \psi(h_1)(n_2)), h_1 h_2)  \\
                                         &= (f(n_1) \cdot (f\circ\psi(h_1))(n_2), h_1 h_2) \\
                                         &= (f(n_1) \cdot (f \circ \psi(h_1) \circ f^{-1} \circ f) (n_2), h_1 h_2) \\
                                         &= (f(n_1) \cdot (\hat{f}(\psi(h_1)) \circ f) (n_2), h_1 h_2) \\
                                         &= (f(n_1) \cdot (\hat{f} \circ \psi)(h_1) f(n_2), h_1 h_2) \\
                                         &= (f(n_1), h_1) (f(n_2), h_2)\\
                                         &= \theta(n_1, h_1) \theta(n_2, h_2).
    \end{align}
    
    Con esto hemos demostrado que $\theta$ es homomorfismo pues abre multiplicaciones y manda inversos en 				
    inversos.\par\null
    
    De manera análoga vemos que $\iota : N \rtimes_{\hat{f} \circ \psi} \rightarrow N \rtimes_\psi H $ 
    definida por $\iota(n,h) = (f^{-1}(n), h)$ es homomorfismo.\par\null
    
    Ahora notemos que:
    
    \begin{align}
        (\iota \circ \theta) (n, h) = \iota(f(n), h) = (f^{-1}(f(n)), h) = (n,h).
    \end{align}		
    
    Por lo tanto $\iota \circ \theta = id_{N \rtimes_\psi H}$.\par\null
    
    An\'alogamente $\theta \circ \iota = id_{N \rtimes_{\hat{f} \circ \psi} H}$. Por lo tanto, $\theta$ y 
    $\iota$ son isomorfismos y con esto terminamos la demostración.
\end{proof}